\documentclass[10.5pt]{article}

% Packages:
\usepackage[utf8]{inputenc}
\usepackage[english]{babel}
\usepackage{float}
\usepackage{graphicx}
\usepackage{amssymb}
\usepackage{pgfplots}
\usepackage{bm}
\usepackage{mathtools}          %loads amsmath as well
\usepackage{titling}
\usepackage{biblatex}
\DeclareGraphicsRule{.tif}{png}{.png}{`convert #1 `dirname #1`/`basename #1 .tif`.png}

% Move title up:
\setlength{\droptitle}{-10em}

\begin{document}

% Title information:
\title{YSP Introductory Material}
\maketitle

    % section introducing cell averaged data
    \section{Data Averages}
    Let's suppose we are looking at data on an interval $x \in [0,1]$, where
    $x$ is the coordinate along this interval. Suppose there is a function
    $u(x)$ defined on this interval. This function is discretized by cutting the
    interval into a number of cells, $N$.
    \begin{figure}[H]
        \center
        \begin{tikzpicture}[sibling distance=2em,every node/.style = {shape=rectangle,align=center,top color=white}]

    % variables
    \def\y{0.0}

    % draw amr blocks
    \draw [thick] (0,\y) rectangle (8,\y);

    % tick marks
    \draw [thick] (0,\y-0.1) rectangle (0,\y+0.1);
    \draw [thick] (8,\y-0.1) rectangle (8,\y+0.1);

    \draw [thick] (1,\y-0.1) rectangle (1,\y+0.1);
    \draw [thick] (2,\y-0.1) rectangle (2,\y+0.1);
    \draw [thick] (3,\y-0.1) rectangle (3,\y+0.1);
    \draw [thick] (4,\y-0.1) rectangle (4,\y+0.1);
    \draw [thick] (5,\y-0.1) rectangle (5,\y+0.1);
    \draw [thick] (6,\y-0.1) rectangle (6,\y+0.1);
    \draw [thick] (7,\y-0.1) rectangle (7,\y+0.1);

\end{tikzpicture}

        \caption{Interval with $N=8$.}
    \end{figure}
    We denote the centers of each cell as $x_{i}$, and the left and right
    boundaries as $x_{i-1/2}$ and $x_{i+1/2}$, respectively.  We define the
    width of each cell as $\triangle x = \frac{1}{N} = x_{i+1/2} - x_{i-1/2}$. For each cell we
    calculate the average amount of $u$ in each interval, and we denote these
    averages as $\overline{u}_{i}$. This is done by perfoming integration of
    $u(x)$ within the cell as
    \begin{equation}
        \overline{u}_{i} = \frac{1}{\triangle x} \int_{x_{i-1/2}}^{x_{i+1/2}}
        u(x) dx.
    \end{equation}

    % section introducing data compression
    \section{Wavelet Transform}
    Now let's imagine that we have a second grid measuring the same function,
    but with half the number of intervals ($N=4$).
    \begin{figure}[H]
        \center
        \begin{tikzpicture}[sibling distance=2em,every node/.style = {shape=rectangle,align=center,top color=white}]

    % variables
    \def\y{0.0}
    \def\yy{-1.25}

    % draw amr blocks
    \draw [thick] (0,\y) rectangle (8,\y);
    \draw [thick] (0,\yy) rectangle (8,\yy);

    % tick marks
    \draw [thick] (0,\y-0.1) rectangle (0,\y+0.1);
    \draw [thick] (8,\y-0.1) rectangle (8,\y+0.1);
    \draw [thick] (1,\y-0.1) rectangle (1,\y+0.1);
    \draw [thick] (2,\y-0.1) rectangle (2,\y+0.1);
    \draw [thick] (3,\y-0.1) rectangle (3,\y+0.1);
    \draw [thick] (4,\y-0.1) rectangle (4,\y+0.1);
    \draw [thick] (5,\y-0.1) rectangle (5,\y+0.1);
    \draw [thick] (6,\y-0.1) rectangle (6,\y+0.1);
    \draw [thick] (7,\y-0.1) rectangle (7,\y+0.1);

    \draw [thick] (0,\yy-0.1) rectangle (0,\yy+0.1);
    \draw [thick] (8,\yy-0.1) rectangle (8,\yy+0.1);
    \draw [thick] (2,\yy-0.1) rectangle (2,\yy+0.1);
    \draw [thick] (4,\yy-0.1) rectangle (4,\yy+0.1);
    \draw [thick] (6,\yy-0.1) rectangle (6,\yy+0.1);

    % labels
    \node [shape=rectangle,align=center] at (0.5,\y-0.5) {$x_{1}$};
    \node [shape=rectangle,align=center] at (1.5,\y-0.5) {$x_{2}$};
    \node [shape=rectangle,align=center] at (2.5,\y-0.5) {$x_{3}$};
    \node [shape=rectangle,align=center] at (3.5,\y-0.5) {$x_{4}$};
    \node [shape=rectangle,align=center] at (4.5,\y-0.5) {$x_{5}$};
    \node [shape=rectangle,align=center] at (5.5,\y-0.5) {$x_{6}$};
    \node [shape=rectangle,align=center] at (6.5,\y-0.5) {$x_{7}$};
    \node [shape=rectangle,align=center] at (7.5,\y-0.5) {$x_{8}$};

    \node [shape=rectangle,align=center] at (1,\yy-0.5) {$x_{1}$};
    \node [shape=rectangle,align=center] at (3,\yy-0.5) {$x_{2}$};
    \node [shape=rectangle,align=center] at (5,\yy-0.5) {$x_{3}$};
    \node [shape=rectangle,align=center] at (7,\yy-0.5) {$x_{4}$};

\end{tikzpicture}

        \caption{Fine grid with $N=8$ and coarse grid with $N=4$.}
    \end{figure}
    Note that each cell on the coarse level has two child cells which live on
    the fine grid. We can \textit{predict} the child values on the finer grid by
    creating an interpolating polynomial based on coarse grid values. Denote the
    coarse cells by $u_{i}^{c}$ and the fine cells by $u_{i}^{f}$. The
    prediction is made with the following interpolation
    \begin{equation}
        \tilde{u}^{f}_{2i} = u^{c}_{i} - \frac{1}{8} \left( u^{c}_{i-1} -
        u^{c}_{i+1} \right) 
    \end{equation}
    Then we want to compute the difference between this polynomial
    approximation, and the true value that we know. We compute the difference
    for each coarse cell,
    \begin{equation}
        d^{c}_{i} = u^{f}_{2i} - \tilde{u}^{f}_{2i}.
    \end{equation}
    Note that near the boundaries we need to use biased interpolation. On the
    left boundary compute
    \begin{equation}
        \tilde{u}^{f}_{2i} = \frac{5}{8} u^{c}_{i} + \frac{1}{2} u^{c}_{i+1}
        -\frac{1}{8} u^{c}_{i+2}.
    \end{equation}
    Near the right boundary compute
    \begin{equation}
        \tilde{u}^{f}_{2i} = \frac{11}{8} u^{c}_{i} - \frac{1}{2} u^{c}_{i-1}
        + \frac{1}{8} u^{c}_{i-2}.
    \end{equation}

\end{document}
