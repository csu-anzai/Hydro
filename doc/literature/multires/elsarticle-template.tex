\documentclass[review]{elsarticle}

\usepackage{lineno,hyperref}
\modulolinenumbers[5]

\journal{Journal of \LaTeX\ Templates}

%%%%%%%%%%%%%%%%%%%%%%%
%% Elsevier bibliography styles
%%%%%%%%%%%%%%%%%%%%%%%
%% To change the style, put a % in front of the second line of the current style and
%% remove the % from the second line of the style you would like to use.
%%%%%%%%%%%%%%%%%%%%%%%

%% Numbered
%\bibliographystyle{model1-num-names}

%% Numbered without titles
%\bibliographystyle{model1a-num-names}

%% Harvard
%\bibliographystyle{model2-names.bst}\biboptions{authoryear}

%% Vancouver numbered
%\usepackage{numcompress}\bibliographystyle{model3-num-names}

%% Vancouver name/year
%\usepackage{numcompress}\bibliographystyle{model4-names}\biboptions{authoryear}

%% APA style
%\bibliographystyle{model5-names}\biboptions{authoryear}

%% AMA style
%\usepackage{numcompress}\bibliographystyle{model6-num-names}

%% `Elsevier LaTeX' style
\bibliographystyle{elsarticle-num}
%%%%%%%%%%%%%%%%%%%%%%%

\begin{document}

\begin{frontmatter}

\title{A multiresolution scheme for adaptive computations on block-structured AMR meshes: applications to reactive flows}
\tnotetext[mytitlenote]{Fully documented templates are available in the elsarticle package on \href{http://www.ctan.org/tex-archive/macros/latex/contrib/elsarticle}{CTAN}.}

%% Group authors per affiliation:
\author{Brandon Gusto\fnref{myfootnote}}
\address{Radarweg 29, Amsterdam}
\fntext[myfootnote]{Since 1880.}

%% or include affiliations in footnotes:
\author[mymainaddrass,mysecondaryaddress]{Elsevier Inc}
\ead[url]{bgusto@fsu.edu}

\author[mysecondaryaddress]{Global Customer Service\corref{mycorrespondingauthor}}
\cortext[mycorrespondingauthor]{Corresponding author}
\ead{support@elsevier.com}

\address[mymainaddress]{1600 John F Kennedy Boulevard, Philadelphia}
\address[mysecondaryaddress]{360 Park Avenue South, New York}

\begin{abstract}
    We present a novel block-structured adaptive mesh refinement scheme
    featuring fully adaptive calculation of fluxes and source terms. The scheme
    addresses a major shortcoming of tree-based AMR codes: that the graded
    nature of the AMR mesh yields blocks whose cells are resolved beyond the
    desired error tolerance. To overcome this issue, we introduce a
    multiresolution representation of the solution not only for the purpose of
    grid adaptation but also to identify regions where fluxes and
    reaction-driven source terms can be interpolated from coarser levels. The
    error introduced by this approximation procedure is shown to be of the same
    order as the local truncation error of the reconstruction scheme. Thus the
    rate of convergence of the underlying spatial reconstruction scheme is
    preserved.  Additionally with respect to parallel applications, the
    multiresolution transform and computation of fluxes and sources on adaptive
    blocks is asynchronous, requiring only one synchronization step which is
    equivalent to the filling of ghost cells for each block. The efficiency of
    the scheme is demonstrated for problems in compressible flow and
    reaction-driven combustion.
\end{abstract}

\begin{keyword}
multiresolution \sep adaptive mesh refinement \sep reactive flows
\MSC[2010] 00-01\sep  99-00
\end{keyword}

\end{frontmatter}

\linenumbers

\section{Introduction}

    % paragraph introduces the need for spatially adaptive grids
    Energetic, reacting, and turbulent flows are characterized by disparate
    spatial and temporal length scales. In the compressible regime, flows are
    capable of producing shock waves, resulting in steep gradients and an
    extremely thin discontinuity which propogates throughout the medium.
    Chemical reactions between fuels and oxidizers become a driving force for
    shock waves due to the large release of energy, and increase in static
    pressure. The burning fronts wherein these reactions take place are also
    highly spatially localized.  These features require a level of mesh
    resolution that would make the problems intractable if applied over the
    entire domain. Therefore, efficient simulation of such flows requires a
    fully adaptive strategy, which we present in the following paper.

    % paragraph introducing adaptive mesh refinement as a concept
    Accurately resolving regions of interest in fluid dynamics simulations for
    real-world applications is typically not feasible without introducing a
    non-uniform spatial mesh. Methods which introduce a hierarchy of nested
    grids are generally described as adaptive mesh refinement (AMR) methods.
    First introduced in (berger1984), AMR methods typically rely on estimates
    of the local truncation error (LTE) to determine regions where refinement
    is necessary for solution accuracy. Some more simple strategies may refine
    based on the magnitudes of the solution gradients, or concentration of a
    quantity of interest. While many strategies are possible, there is not yet
    a significant amount of mathematical theory available to quantify the
    solution accuracy for AMR simulations. For a full review of the LTE
    estimators and refinement criterion, readers are referred to BLANK.

    % paragraph reviews the work of harten and multiresolution methods
    Alternate approaches to dynamic grid adaptation based on wavelet theory
    became popular after the seminal papers by Harten \cite{harten1994}, were
    introduced. In this work, a multiresolution representation of the discrete
    solution on a uniform grid was used for adaptively computing the divergence
    of the flux within a finite volume framework. Rather than adapt the grid
    directly, the idea was to accelerate the computation of fluxes using the
    multiresolution information. In this approach, eligible fluxes in
    sufficiently smooth regions are interpolated from fluxes obtained at
    interfaces corresponding to coarser grid levels. The original scheme was
    applied solely to hyperbolic conservation laws in one spatial dimension,
    but was then expanded by Bihari et. al. to two-dimensional simulations in,
    followed by the inclusion of viscous terms in, and then to source terms in
    the context of reactive flows in (bihari). These works retained the
    original spirit of Harten's scheme, which was to evolve the solution on a
    uniform grid, but use multiresolution information to identify regions where
    flux (and/or source term) computations may be avoided.

    % talk about block-structured AMR
    Regarding the implementation of AMR methods on large networks of parallel
    computers, certain engineering realities have neccessitated the reduction in
    granularity of the adaptive refinement. To use a single computational cell
    as the unit for refinement (i.e. cell-based refinement) introduces a number
    of costly compromises. Firstly, such an adapted grid requires the
    reconstruction method of choice to utilize nonuniform stencils, requiring
    increased computational resources. More significantly, the cell-based
    refinement requires costly data traversal. Traversing tree space requires on
    average $\mathcal{O}(n^d)$ operations, where $n$ is the
    number of cells per dimension, $d$. Thus most AMR codes make use of some
    type of block-based approach. Tree-based block-structured codes, where each
    block consists of a fixed number of cells, are a very popular choice. These
    types of approaches are implemented in a number of AMR libraries including
    Paramesh, p4est, and others. This approach allows for
    simple mesh management procedures, and scales well for very large numbers of
    processors in parallel.  One clear drawback however is the gradedness of the
    tree, which necessitates that no branch can have an incomplete set of
    children. This typically leads to refinement of many blocks which would not
    be otherwise flagged by refinement indicators. A further complication
    imposed by most finite volume solvers is that there can not be a jump in
    refinement greater than one level. Together these consequences of AMR
    represent a non-trivial decrease in performance due to the fact that the
    mesh is not optimal (in some sense).

    % review the multiresolution-adaptive papers
    Although Harten's original scheme was intended to be an alternative to
    spatially non-uniform grid adaptation, a series of papers have since
    reintroduced the concept of non-uniform grids within the MR framework. Thus
    the AMR approach was essentially redeveloped but with the refinement
    criterion defined by the MR representation rather than with the traditional
    metrics mentioned previously. The first fully adaptive scheme was presented
    by Cohen et. al. to study hyperbolic conservation laws in two dimensions in
    More recently, Rossinilli et. al. explored the use of
    wavelet-based refinement indicators for block-based adaptation.

    % punchline
    In the following work, we present a novel block-based adaptive mesh
    refinement scheme using wavelet-based indicators, where Harten's scheme is
    used to essentially treat each block as a uniform grid.  The indicators are
    used for two purposes: (1) refinement, and (2) adaptively calculating fluxes
    and source terms. We show how this scheme costs essentially nothing in
    additional computation (the MR information is recycled after adaptation),
    while not impacting the accuracy of the solver.

    In the section 2 of this paper, we describe the conservations laws of
    interest as well as the underlying finite volume method and discretization.
    In section 3, we introduce the multiresolution decomposition of the
    numerical solution. In section 4, the block-based adaptive scheme is
    introduced, which combines grid adaptation with the interpolatory scheme of
    Harten.

\section{Bibliography styles}

There are various bibliography styles available. You can select the style of your choice in the preamble of this document. These styles are Elsevier styles based on standard styles like Harvard and Vancouver. Please use Bib\TeX\ to generate your bibliography and include DOIs whenever available.

Here are two sample references: \cite{Feynman1963118,Dirac1953888}.

\section*{References}

\bibliography{mybibfile}

\end{document}
