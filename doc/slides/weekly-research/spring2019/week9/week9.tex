\documentclass{beamer}
\usepackage{graphicx,amsmath,amsfonts,amssymb,listings,tikz}
\usepackage{multimedia}
\usetheme{Montpellier}
\usecolortheme{beaver}
\beamerdefaultoverlayspecification{<+->}

% user commands
\newcommand{\weeknum}{9}

\begin{document}

\section{Group Meeting}
\title{Group Meeting \\ Week \weeknum, Spring 2019}
\author{Brandon Gusto} %
\institute{Dept. of Scientific Computing \\ Florida State University}
\date{\today}
\frame{\titlepage}

\section{Multiresolution Scheme}

\begin{frame}{Hybrid Wavelet/AMR Scheme}
    \begin{columns}
        \begin{column}{0.58\textwidth}
            The goal of this project is to marry the benefits of wavelet analysis with the more developed AMR strategies:
            \begin{itemize}
                \item can wavelet sensors augment LTE for refinement?
                \item can regularity information reduce computational expense on fine grids?
            \end{itemize}
        \end{column}
        \begin{column}{0.44\textwidth}
            \scalebox{0.4}{
                \input{plots/detail_j3.tex}
            }
        \end{column}
    \end{columns}
\end{frame}

\begin{frame}{Harten's MR Scheme}
    The one-dimensional reference system of conservation laws is
    \begin{equation*}
        \mathbf{U}_{t} + \mathbf{F}(\mathbf{U})_{x} = 0,
    \end{equation*}
    where $\mathbf{U} = (\rho,\rho u,E)$ is a vector of conserved quantities.
    In semi-discrete form,
    \begin{equation*}
        \frac{\partial \mathbf{U}_{i}}{\partial t} = -\frac{1}{h} \left( \mathbf{F}_{i+\frac{1}{2}}
            - \mathbf{F}_{i-\frac{1}{2}} \right) = \mathbf{R}_{i}
    \end{equation*}
    where the $i$ denotes spatial index, and $\mathbf{R}$ is residual.
\end{frame}

\begin{frame}{Harten's MR Scheme}
    Define multiple, nested grids
    \begin{equation*}
        \mathbf{G}^{l} = \left\{ x^{l}_{i+\frac{1}{2}} \right\}_{i=0}^{N_{l}} =
            \left\{ x^{l+1}_{i+\frac{1}{2}} \right\}_{i=0,\text{i even}}^{N^{l+1}}.
    \end{equation*}
    Coarsening of a cell done via
    \begin{equation*}
        \mathbf{U}^{l}_{i} = \frac{1}{2} \left( \mathbf{U}^{l+1}_{2i} + \mathbf{U}^{l+1}_{2i+1} \right)
    \end{equation*}
    and the prediction from coarse to fine is
    \begin{equation*}
        \mathbf{\hat{U}}^{l+1}_{2i+1} = \sum_{j=1-s}^{s-1} \gamma_{l} \mathbf{U}^{l}_{i+l}
    \end{equation*}
\end{frame}

\begin{frame}{Harten's MR Scheme}
\end{document}
