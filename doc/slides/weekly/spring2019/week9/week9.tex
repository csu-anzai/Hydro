\documentclass{beamer}
\usepackage{graphicx,amsmath,amsfonts,amssymb,listings,tikz}
\usepackage{multimedia}
\usetheme{Montpellier}
\usecolortheme{beaver}
\beamerdefaultoverlayspecification{<+->}

% user commands
\newcommand{\weeknum}{9}

\begin{document}

\section{Group Meeting}
\title{Group Meeting \\ Week \weeknum, Spring 2019}
\author{Brandon Gusto} %
\institute{Dept. of Scientific Computing \\ Florida State University}
\date{\today}
\frame{\titlepage}

\section{Multiresolution Scheme}

\begin{frame}{Hybrid Wavelet/AMR Scheme}
    \begin{columns}
        \begin{column}{0.58\textwidth}
            The goal of this project is to marry the benefits of wavelet analysis with the more developed AMR strategies:
            \begin{itemize}
                \item can wavelet sensors augment LTE for refinement?
                \item can regularity information reduce computational expense on fine grids?
            \end{itemize}
        \end{column}
        \begin{column}{0.44\textwidth}
            \scalebox{0.4}{
                % GNUPLOT: LaTeX picture
\setlength{\unitlength}{0.240900pt}
\ifx\plotpoint\undefined\newsavebox{\plotpoint}\fi
\sbox{\plotpoint}{\rule[-0.200pt]{0.400pt}{0.400pt}}%
\begin{picture}(1500,900)(0,0)
\sbox{\plotpoint}{\rule[-0.200pt]{0.400pt}{0.400pt}}%
\put(171.0,131.0){\rule[-0.200pt]{4.818pt}{0.400pt}}
\put(151,131){\makebox(0,0)[r]{$-0.5$}}
\put(1419.0,131.0){\rule[-0.200pt]{4.818pt}{0.400pt}}
\put(171.0,277.0){\rule[-0.200pt]{4.818pt}{0.400pt}}
\put(151,277){\makebox(0,0)[r]{$0$}}
\put(1419.0,277.0){\rule[-0.200pt]{4.818pt}{0.400pt}}
\put(171.0,422.0){\rule[-0.200pt]{4.818pt}{0.400pt}}
\put(151,422){\makebox(0,0)[r]{$0.5$}}
\put(1419.0,422.0){\rule[-0.200pt]{4.818pt}{0.400pt}}
\put(171.0,568.0){\rule[-0.200pt]{4.818pt}{0.400pt}}
\put(151,568){\makebox(0,0)[r]{$1$}}
\put(1419.0,568.0){\rule[-0.200pt]{4.818pt}{0.400pt}}
\put(171.0,713.0){\rule[-0.200pt]{4.818pt}{0.400pt}}
\put(151,713){\makebox(0,0)[r]{$1.5$}}
\put(1419.0,713.0){\rule[-0.200pt]{4.818pt}{0.400pt}}
\put(171.0,859.0){\rule[-0.200pt]{4.818pt}{0.400pt}}
\put(151,859){\makebox(0,0)[r]{$2$}}
\put(1419.0,859.0){\rule[-0.200pt]{4.818pt}{0.400pt}}
\put(171.0,131.0){\rule[-0.200pt]{0.400pt}{4.818pt}}
\put(171,90){\makebox(0,0){$-4$}}
\put(171.0,839.0){\rule[-0.200pt]{0.400pt}{4.818pt}}
\put(330.0,131.0){\rule[-0.200pt]{0.400pt}{4.818pt}}
\put(330,90){\makebox(0,0){$-3$}}
\put(330.0,839.0){\rule[-0.200pt]{0.400pt}{4.818pt}}
\put(488.0,131.0){\rule[-0.200pt]{0.400pt}{4.818pt}}
\put(488,90){\makebox(0,0){$-2$}}
\put(488.0,839.0){\rule[-0.200pt]{0.400pt}{4.818pt}}
\put(647.0,131.0){\rule[-0.200pt]{0.400pt}{4.818pt}}
\put(647,90){\makebox(0,0){$-1$}}
\put(647.0,839.0){\rule[-0.200pt]{0.400pt}{4.818pt}}
\put(805.0,131.0){\rule[-0.200pt]{0.400pt}{4.818pt}}
\put(805,90){\makebox(0,0){$0$}}
\put(805.0,839.0){\rule[-0.200pt]{0.400pt}{4.818pt}}
\put(964.0,131.0){\rule[-0.200pt]{0.400pt}{4.818pt}}
\put(964,90){\makebox(0,0){$1$}}
\put(964.0,839.0){\rule[-0.200pt]{0.400pt}{4.818pt}}
\put(1122.0,131.0){\rule[-0.200pt]{0.400pt}{4.818pt}}
\put(1122,90){\makebox(0,0){$2$}}
\put(1122.0,839.0){\rule[-0.200pt]{0.400pt}{4.818pt}}
\put(1281.0,131.0){\rule[-0.200pt]{0.400pt}{4.818pt}}
\put(1281,90){\makebox(0,0){$3$}}
\put(1281.0,839.0){\rule[-0.200pt]{0.400pt}{4.818pt}}
\put(1439.0,131.0){\rule[-0.200pt]{0.400pt}{4.818pt}}
\put(1439,90){\makebox(0,0){$4$}}
\put(1439.0,839.0){\rule[-0.200pt]{0.400pt}{4.818pt}}
\put(171.0,131.0){\rule[-0.200pt]{0.400pt}{175.375pt}}
\put(171.0,131.0){\rule[-0.200pt]{305.461pt}{0.400pt}}
\put(1439.0,131.0){\rule[-0.200pt]{0.400pt}{175.375pt}}
\put(171.0,859.0){\rule[-0.200pt]{305.461pt}{0.400pt}}
\put(30,495){\makebox(0,0){$\psi(x)$}}
\put(805,29){\makebox(0,0){$x$}}
\put(171,277){\usebox{\plotpoint}}
\put(706,275.67){\rule{1.204pt}{0.400pt}}
\multiput(706.00,276.17)(2.500,-1.000){2}{\rule{0.602pt}{0.400pt}}
\put(171.0,277.0){\rule[-0.200pt]{128.881pt}{0.400pt}}
\put(721,275.67){\rule{1.204pt}{0.400pt}}
\multiput(721.00,275.17)(2.500,1.000){2}{\rule{0.602pt}{0.400pt}}
\put(726,276.67){\rule{1.204pt}{0.400pt}}
\multiput(726.00,276.17)(2.500,1.000){2}{\rule{0.602pt}{0.400pt}}
\put(731,278.17){\rule{1.100pt}{0.400pt}}
\multiput(731.00,277.17)(2.717,2.000){2}{\rule{0.550pt}{0.400pt}}
\put(736,280.17){\rule{1.100pt}{0.400pt}}
\multiput(736.00,279.17)(2.717,2.000){2}{\rule{0.550pt}{0.400pt}}
\put(741,281.67){\rule{1.204pt}{0.400pt}}
\multiput(741.00,281.17)(2.500,1.000){2}{\rule{0.602pt}{0.400pt}}
\multiput(746.00,283.61)(0.909,0.447){3}{\rule{0.767pt}{0.108pt}}
\multiput(746.00,282.17)(3.409,3.000){2}{\rule{0.383pt}{0.400pt}}
\put(711.0,276.0){\rule[-0.200pt]{2.409pt}{0.400pt}}
\put(755,284.17){\rule{1.100pt}{0.400pt}}
\multiput(755.00,285.17)(2.717,-2.000){2}{\rule{0.550pt}{0.400pt}}
\multiput(760.59,281.26)(0.477,-0.710){7}{\rule{0.115pt}{0.660pt}}
\multiput(759.17,282.63)(5.000,-5.630){2}{\rule{0.400pt}{0.330pt}}
\multiput(765.59,271.94)(0.477,-1.489){7}{\rule{0.115pt}{1.220pt}}
\multiput(764.17,274.47)(5.000,-11.468){2}{\rule{0.400pt}{0.610pt}}
\multiput(770.59,257.27)(0.477,-1.712){7}{\rule{0.115pt}{1.380pt}}
\multiput(769.17,260.14)(5.000,-13.136){2}{\rule{0.400pt}{0.690pt}}
\multiput(775.59,241.60)(0.477,-1.601){7}{\rule{0.115pt}{1.300pt}}
\multiput(774.17,244.30)(5.000,-12.302){2}{\rule{0.400pt}{0.650pt}}
\multiput(780.59,227.60)(0.477,-1.267){7}{\rule{0.115pt}{1.060pt}}
\multiput(779.17,229.80)(5.000,-9.800){2}{\rule{0.400pt}{0.530pt}}
\multiput(785.59,217.26)(0.477,-0.710){7}{\rule{0.115pt}{0.660pt}}
\multiput(784.17,218.63)(5.000,-5.630){2}{\rule{0.400pt}{0.330pt}}
\multiput(790.00,213.60)(0.627,0.468){5}{\rule{0.600pt}{0.113pt}}
\multiput(790.00,212.17)(3.755,4.000){2}{\rule{0.300pt}{0.400pt}}
\multiput(795.59,217.00)(0.477,2.046){7}{\rule{0.115pt}{1.620pt}}
\multiput(794.17,217.00)(5.000,15.638){2}{\rule{0.400pt}{0.810pt}}
\multiput(800.59,236.00)(0.477,4.495){7}{\rule{0.115pt}{3.380pt}}
\multiput(799.17,236.00)(5.000,33.985){2}{\rule{0.400pt}{1.690pt}}
\multiput(805.59,277.00)(0.477,7.389){7}{\rule{0.115pt}{5.460pt}}
\multiput(804.17,277.00)(5.000,55.667){2}{\rule{0.400pt}{2.730pt}}
\multiput(810.59,344.00)(0.477,9.393){7}{\rule{0.115pt}{6.900pt}}
\multiput(809.17,344.00)(5.000,70.679){2}{\rule{0.400pt}{3.450pt}}
\multiput(815.59,429.00)(0.477,10.395){7}{\rule{0.115pt}{7.620pt}}
\multiput(814.17,429.00)(5.000,78.184){2}{\rule{0.400pt}{3.810pt}}
\multiput(820.59,523.00)(0.477,10.506){7}{\rule{0.115pt}{7.700pt}}
\multiput(819.17,523.00)(5.000,79.018){2}{\rule{0.400pt}{3.850pt}}
\multiput(825.59,618.00)(0.477,9.949){7}{\rule{0.115pt}{7.300pt}}
\multiput(824.17,618.00)(5.000,74.848){2}{\rule{0.400pt}{3.650pt}}
\multiput(830.59,708.00)(0.477,8.391){7}{\rule{0.115pt}{6.180pt}}
\multiput(829.17,708.00)(5.000,63.173){2}{\rule{0.400pt}{3.090pt}}
\multiput(835.59,784.00)(0.477,5.942){7}{\rule{0.115pt}{4.420pt}}
\multiput(834.17,784.00)(5.000,44.826){2}{\rule{0.400pt}{2.210pt}}
\multiput(840.59,838.00)(0.477,2.269){7}{\rule{0.115pt}{1.780pt}}
\multiput(839.17,838.00)(5.000,17.306){2}{\rule{0.400pt}{0.890pt}}
\multiput(845.59,851.61)(0.477,-2.269){7}{\rule{0.115pt}{1.780pt}}
\multiput(844.17,855.31)(5.000,-17.306){2}{\rule{0.400pt}{0.890pt}}
\multiput(850.59,819.65)(0.477,-5.942){7}{\rule{0.115pt}{4.420pt}}
\multiput(849.17,828.83)(5.000,-44.826){2}{\rule{0.400pt}{2.210pt}}
\multiput(855.60,752.04)(0.468,-11.009){5}{\rule{0.113pt}{7.700pt}}
\multiput(854.17,768.02)(4.000,-60.018){2}{\rule{0.400pt}{3.850pt}}
\multiput(859.59,677.70)(0.477,-9.949){7}{\rule{0.115pt}{7.300pt}}
\multiput(858.17,692.85)(5.000,-74.848){2}{\rule{0.400pt}{3.650pt}}
\multiput(864.59,586.04)(0.477,-10.506){7}{\rule{0.115pt}{7.700pt}}
\multiput(863.17,602.02)(5.000,-79.018){2}{\rule{0.400pt}{3.850pt}}
\multiput(869.59,491.37)(0.477,-10.395){7}{\rule{0.115pt}{7.620pt}}
\multiput(868.17,507.18)(5.000,-78.184){2}{\rule{0.400pt}{3.810pt}}
\multiput(874.59,400.36)(0.477,-9.393){7}{\rule{0.115pt}{6.900pt}}
\multiput(873.17,414.68)(5.000,-70.679){2}{\rule{0.400pt}{3.450pt}}
\multiput(879.59,321.33)(0.477,-7.389){7}{\rule{0.115pt}{5.460pt}}
\multiput(878.17,332.67)(5.000,-55.667){2}{\rule{0.400pt}{2.730pt}}
\multiput(884.59,262.97)(0.477,-4.495){7}{\rule{0.115pt}{3.380pt}}
\multiput(883.17,269.98)(5.000,-33.985){2}{\rule{0.400pt}{1.690pt}}
\multiput(889.59,229.28)(0.477,-2.046){7}{\rule{0.115pt}{1.620pt}}
\multiput(888.17,232.64)(5.000,-15.638){2}{\rule{0.400pt}{0.810pt}}
\multiput(894.00,215.94)(0.627,-0.468){5}{\rule{0.600pt}{0.113pt}}
\multiput(894.00,216.17)(3.755,-4.000){2}{\rule{0.300pt}{0.400pt}}
\multiput(899.59,213.00)(0.477,0.710){7}{\rule{0.115pt}{0.660pt}}
\multiput(898.17,213.00)(5.000,5.630){2}{\rule{0.400pt}{0.330pt}}
\multiput(904.59,220.00)(0.477,1.267){7}{\rule{0.115pt}{1.060pt}}
\multiput(903.17,220.00)(5.000,9.800){2}{\rule{0.400pt}{0.530pt}}
\multiput(909.59,232.00)(0.477,1.601){7}{\rule{0.115pt}{1.300pt}}
\multiput(908.17,232.00)(5.000,12.302){2}{\rule{0.400pt}{0.650pt}}
\multiput(914.59,247.00)(0.477,1.712){7}{\rule{0.115pt}{1.380pt}}
\multiput(913.17,247.00)(5.000,13.136){2}{\rule{0.400pt}{0.690pt}}
\multiput(919.59,263.00)(0.477,1.489){7}{\rule{0.115pt}{1.220pt}}
\multiput(918.17,263.00)(5.000,11.468){2}{\rule{0.400pt}{0.610pt}}
\multiput(924.59,277.00)(0.477,0.710){7}{\rule{0.115pt}{0.660pt}}
\multiput(923.17,277.00)(5.000,5.630){2}{\rule{0.400pt}{0.330pt}}
\put(929,284.17){\rule{1.100pt}{0.400pt}}
\multiput(929.00,283.17)(2.717,2.000){2}{\rule{0.550pt}{0.400pt}}
\put(751.0,286.0){\rule[-0.200pt]{0.964pt}{0.400pt}}
\multiput(939.00,284.95)(0.909,-0.447){3}{\rule{0.767pt}{0.108pt}}
\multiput(939.00,285.17)(3.409,-3.000){2}{\rule{0.383pt}{0.400pt}}
\put(944,281.67){\rule{1.204pt}{0.400pt}}
\multiput(944.00,282.17)(2.500,-1.000){2}{\rule{0.602pt}{0.400pt}}
\put(949,280.17){\rule{1.100pt}{0.400pt}}
\multiput(949.00,281.17)(2.717,-2.000){2}{\rule{0.550pt}{0.400pt}}
\put(954,278.17){\rule{1.100pt}{0.400pt}}
\multiput(954.00,279.17)(2.717,-2.000){2}{\rule{0.550pt}{0.400pt}}
\put(959,276.67){\rule{1.204pt}{0.400pt}}
\multiput(959.00,277.17)(2.500,-1.000){2}{\rule{0.602pt}{0.400pt}}
\put(964,275.67){\rule{0.964pt}{0.400pt}}
\multiput(964.00,276.17)(2.000,-1.000){2}{\rule{0.482pt}{0.400pt}}
\put(934.0,286.0){\rule[-0.200pt]{1.204pt}{0.400pt}}
\put(978,275.67){\rule{1.204pt}{0.400pt}}
\multiput(978.00,275.17)(2.500,1.000){2}{\rule{0.602pt}{0.400pt}}
\put(968.0,276.0){\rule[-0.200pt]{2.409pt}{0.400pt}}
\put(983.0,277.0){\rule[-0.200pt]{109.850pt}{0.400pt}}
\put(171.0,131.0){\rule[-0.200pt]{0.400pt}{175.375pt}}
\put(171.0,131.0){\rule[-0.200pt]{305.461pt}{0.400pt}}
\put(1439.0,131.0){\rule[-0.200pt]{0.400pt}{175.375pt}}
\put(171.0,859.0){\rule[-0.200pt]{305.461pt}{0.400pt}}
\end{picture}

            }
        \end{column}
    \end{columns}
\end{frame}

\begin{frame}{Harten's MR Scheme}
    The one-dimensional reference system of conservation laws is
    \begin{equation*}
        \mathbf{U}_{t} + \mathbf{F}(\mathbf{U})_{x} = 0,
    \end{equation*}
    where $\mathbf{U} = (\rho,\rho u,E)$ is a vector of conserved quantities.
    In semi-discrete form,
    \begin{equation*}
        \frac{\partial \mathbf{U}_{i}}{\partial t} = -\frac{1}{h} \left( \mathbf{F}_{i+\frac{1}{2}}
            - \mathbf{F}_{i-\frac{1}{2}} \right) = \mathbf{R}_{i}
    \end{equation*}
    where the $i$ denotes spatial index, and $\mathbf{R}$ is residual.
\end{frame}

\begin{frame}{Harten's MR Scheme}
    Define multiple, nested grids
    \begin{equation*}
        \mathbf{G}^{l} = \left\{ x^{l}_{i+\frac{1}{2}} \right\}_{i=0}^{N_{l}} =
            \left\{ x^{l+1}_{i+\frac{1}{2}} \right\}_{i=0,\text{i even}}^{N^{l+1}}.
    \end{equation*}
    Coarsening of a cell done via
    \begin{equation*}
        \mathbf{U}^{l}_{i} = \frac{1}{2} \left( \mathbf{U}^{l+1}_{2i} + \mathbf{U}^{l+1}_{2i+1} \right)
    \end{equation*}
    and the prediction from coarse to fine is
    \begin{equation*}
        \mathbf{\hat{U}}^{l+1}_{2i+1} = \sum_{j=1-s}^{s-1} \gamma_{l} \mathbf{U}^{l}_{i+l}
    \end{equation*}
\end{frame}

\begin{frame}{Harten's MR Scheme}
\end{document}
