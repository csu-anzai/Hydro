\documentclass[a4paper,10pt]{article}
\usepackage[utf8x]{inputenc}
\usepackage{amsmath,amsbsy}

%opening
\title{Computing the volume of solids defined by polyhedral meshes in orthogonal coordinate systems}
\author{Timothy Handy}

\begin{document}

\maketitle

\begin{abstract}

\end{abstract}

\section{Motivation}

Consider an n-dimensional mesh with positions given by an orthogonal coordinate system (OSC). Any orthogonal coordinate system can be viewed as a uniform Cartesian coordinate system (or logically Cartesian coordinate system LCCS) with an appropriate scaling field, or metric, which decribes how geometric properties (length, area, volume) change in this Cartesian space.

Returning to our mesh, every cell is an orthogonal, axis aligned brick in the LCCS. If we arbitrarily truncate or clip this cell, we will obviously have a change in the geometry when viewing the result in its native coordinate system.

\subsection{Example}
Consider the LCCS $r-\theta-z$ for cylindrical coordinates. Fixing $r$ and $z$ produces circles as $\theta$ varies. If we consider the domain $r_{in}\le r\le r_{out}$, $0\le\theta\le2\pi$, and $z=0$, we produce a filled ring. The area of this ring is given by 

\begin{equation}
	A_{ring} = \pi\left(r_{out}^2 - r_{in}^2\right)
\end{equation}

If we naively attempt to calculate the area of the ring as 

\begin{equation}
A_{ring}^{LCCS} = \left(r_{out}-r_{in}\right)\left(\theta_{max}-\theta_{min}\right)
\end{equation}

we see that

\begin{equation}
	\frac{A_{ring}}{A_{ring}^{LCCS}} = \pi\frac{r_{out} + r_{in}}{\theta_{max}-\theta_{min}} \neq 1.
\end{equation}

This result stresses the importance of including the metric when computing quantities on an LCCS.

\section{Computing Area}

The divergence theorem in a plane is given by 

\begin{equation}
\label{e:divplane}
\int_{S}\nabla\cdot\mathbf{F} dA = \oint_{\partial S} \mathbf{F}\cdot\boldsymbol{\hat{n}} ds	
\end{equation}

where $\mathbf{F}$ is some vector field, $\boldsymbol{\hat{n}}$ is the normal to the curve, and $ds$ is the differential arc length.

The goal is to choose some vector field $\mathbf{F}$ (\emph{defined on the surface $S$}) such that $\nabla\cdot\mathbf{F}=1$, which reduces Eq. (\ref{e:divplane}) to 

\begin{equation}
\label{e:area}
  A_{S} = \int_{S}dA = \oint_{\partial S} \mathbf{F}\cdot\boldsymbol{\hat{n}} ds
\end{equation}

\subsection{Area of a surface in 3-space}

In general, $\partial S$ may be a curve in 3-space, which complicates computation of the normal component. In the case where the curve lies in one of the coordinate planes, the normal is easy to directly calculate. Let $\alpha-\beta-\gamma$ define the axes of a LCCS. In the following we consider the cases where the curve is either in the $\alpha\beta$ plane, the $\alpha\gamma$ plane, the $\beta\gamma$ plane, or along an embedded surface in $\alpha-\beta-\gamma$. For all cases, $\mathbf{N}$ denotes the normal to the surface the curve lies on and may be written as
\begin{equation}
\label{e:N}
	\mathbf{N} = N_{\alpha}(\alpha,\beta,\gamma)\boldsymbol{\hat{\alpha}} + N_{\beta}(\alpha,\beta,\gamma)\boldsymbol{\hat{\beta}} + N_{\gamma}(\alpha,\beta,\gamma)\boldsymbol{\hat{\gamma}}
\end{equation}


Additionally, the general line element can be written as 
\begin{equation}
	d\mathbf{s} = h_{\alpha}(\alpha,\beta,\gamma,d\alpha)\boldsymbol{\hat{\alpha}} + h_{\beta}(\alpha,\beta,\gamma,d\beta)\boldsymbol{\hat{\beta}} + h_{\gamma}(\alpha,\beta,\gamma,d\gamma)\boldsymbol{\hat{\gamma}} 
\end{equation}

From these two statements, the unit normal to the curve is, in general, given by 
\begin{equation}
\label{e:curvenormal}
	\boldsymbol{\hat{n}} = \frac{d\mathbf{s}\times\mathbf{N}}{\left|d\mathbf{s}\times\mathbf{N}\right|}
\end{equation}



\subsubsection{$\alpha-\beta$ plane}

For a curve in the $\alpha\beta$ plane, the surface normal is
\begin{equation}
	\mathbf{N} = 0\boldsymbol{\hat{\alpha}} + 0\boldsymbol{\hat{\beta}} + 1\boldsymbol{\hat{\gamma}}
\end{equation}

Application of Eq.(\ref{e:curvenormal}) gives
\begin{equation}
	\boldsymbol{\hat{n}} = \frac{h_{\beta}\boldsymbol{\hat{\alpha}} - h_{\alpha}\boldsymbol{\hat{\beta}} + 0\boldsymbol{\hat{\gamma}}}{\sqrt{h_{\alpha}^2 + h_{\beta}^2}}
\end{equation}

The differential arc length becomes (with $h_{\gamma}\equiv0$)
\begin{equation}
	ds = \sqrt{h_{\alpha}^2+h_{\beta}^2}
\end{equation}

and Eq.(\ref{e:area}) reduces to
\begin{equation}
	A_{S} = \oint_{\partial S} \mathbf{F}\cdot\left[h_{\beta}, -h_{\alpha}, 0\right]
\end{equation}



%%----------------------------------------------------------------------------------------------------

\subsubsection{$\alpha-\gamma$ plane}

For a curve in the $\alpha\gamma$ plane, the surface normal is
\begin{equation}
	\mathbf{N} = 0\boldsymbol{\hat{\alpha}} + 1\boldsymbol{\hat{\beta}} + 0\boldsymbol{\hat{\gamma}}
\end{equation}

Application of Eq.(\ref{e:curvenormal}) gives
\begin{equation}
	\boldsymbol{\hat{n}} = \frac{-h\boldsymbol{\hat{\alpha}} + 0\boldsymbol{\hat{\beta}} + h_{\alpha}\boldsymbol{\hat{\gamma}}}{\sqrt{h_{\alpha}^2 + h_{\gamma}^2}}
\end{equation}

The differential arc length becomes (with $h_{\beta}\equiv0$)
\begin{equation}
	ds = \sqrt{h_{\alpha}^2+h_{\gamma}^2}
\end{equation}

and Eq.(\ref{e:area}) reduces to
\begin{equation}
	A_{S} = \oint_{\partial S} \mathbf{F}\cdot\left[-h_{\gamma}, 0, h_{\alpha}\right]
\end{equation}

%%----------------------------------------------------------------------------------------------------

\subsubsection{$\beta-\gamma$ plane}

For a curve in the $\beta\gamma$ plane, the surface normal is
\begin{equation}
	\mathbf{N} = 1\boldsymbol{\hat{\alpha}} + 0\boldsymbol{\hat{\beta}} + 0\boldsymbol{\hat{\gamma}}
\end{equation}

Application of Eq.(\ref{e:curvenormal}) gives
\begin{equation}
	\boldsymbol{\hat{n}} = \frac{0\boldsymbol{\hat{\alpha}} + h_{\gamma}\boldsymbol{\hat{\beta}} - h_{\beta}\boldsymbol{\hat{\gamma}}}{\sqrt{h_{\beta}^2 + h_{\gamma}^2}}
\end{equation}

The differential arc length becomes (with $h_{\alpha}\equiv0$)
\begin{equation}
	ds = \sqrt{h_{\beta}^2+h_{\gamma}^2}
\end{equation}

and Eq.(\ref{e:area}) reduces to
\begin{equation}
	A_{S} = \oint_{\partial S} \mathbf{F}\cdot\left[0, h_{\gamma}, -h_{\beta}\right]
\end{equation}

\subsubsection{General surface}

For a curve bounding a general surface embedded in 3-space, the surface normal is given by Eq.(\ref{e:N}).

Application of Eq.(\ref{e:curvenormal}) gives
\begin{equation}
	\boldsymbol{\hat{n}} = \frac{\left( h_{\beta}N_{\gamma} - h_{\gamma}N_{\beta} \right)\boldsymbol{\hat{\alpha}} + \left( h_{\gamma}N_{\alpha} - h_{\alpha}N_{\gamma} \right)\boldsymbol{\hat{\beta}} + \left( h_{\alpha}N_{\beta} - h_{\beta}N_{\alpha} \right)\boldsymbol{\hat{\alpha}}}{\sqrt{\left( h_{\beta}N_{\gamma} - h_{\gamma}N_{\beta} \right)^2 + \left( h_{\gamma}N_{\alpha} - h_{\alpha}N_{\gamma} \right)^2 + \left( h_{\alpha}N_{\beta} - h_{\beta}N_{\alpha} \right)^2}}
\end{equation}

This quickly gets out of control as the denominator does not cancel with the differential arc length ($ds$) and one is left with a mess: 
\begin{equation}
	A_{S} = \oint_{\partial S} \frac{\left(\mathbf{F}\cdot\left[\left(h_{\beta}N_{\gamma} - h_{\gamma}N_{\beta} \right), \left( h_{\gamma}N_{\alpha} - h_{\alpha}N_{\gamma} \right), \left( h_{\alpha}N_{\beta} - h_{\beta}N_{\alpha} \right)\right]\right)\sqrt{h_{\alpha}^2 + h_{\beta}^2 + h_{\gamma}^2}}{\sqrt{\left( h_{\beta}N_{\gamma} - h_{\gamma}N_{\beta} \right)^2 + \left( h_{\gamma}N_{\alpha} - h_{\alpha}N_{\gamma} \right)^2 + \left( h_{\alpha}N_{\beta} - h_{\beta}N_{\alpha} \right)^2}}
\end{equation}



\subsubsection{Cartesian}


The coordinate notation change for the LCCS is given by

\begin{equation}
	\left(\alpha,\beta,\gamma\right)\leftarrow\left(x,y,z\right)
\end{equation}


In Cartesian geometry, the metric is given by 
\begin{align}
& h_{\alpha} = d\alpha\nonumber\\
& h_{\beta} = d\beta\\
& h_{\gamma} = d\gamma\nonumber
\end{align}


and the differential line element is
\begin{equation}
	d\mathbf{s} = d\alpha\boldsymbol{\hat{\alpha}} + d\beta\boldsymbol{\hat{\beta}} + d\gamma\boldsymbol{\hat{\gamma}} 
\end{equation}
which implies a differential arc length of
\begin{equation}
	ds = \sqrt{d\mathbf{s}\cdot d\mathbf{s}} = \sqrt{d\alpha^2 + d\beta^2 + d\gamma^2}
\end{equation}

For planar curves, the area integrals are given as

\begin{table}[h]
	\centering
	\begin{tabular}{|c|c|}
		\hline
		
		plane & area integral \\ \hline 
		$\alpha\beta$ & $A_{\alpha\beta} = \oint_{\partial S} F_{\alpha}d\beta - F_{\beta}d\alpha$ \\ \hline 
		$\alpha\gamma$ & $A_{\alpha\gamma} = \oint_{\partial S} F_{\gamma}d\alpha - F_{\alpha}d\gamma$ \\ \hline 
		$\beta\gamma$ & $A_{\beta\gamma} = \oint_{\partial S} F_{\beta}d\gamma - F_{\gamma}d\beta$ \\ \hline 
		
		\hline
	\end{tabular}
	\caption{Area integrals for planar curves in Cartesian geometry}
	\label{t:cartesian}
\end{table}


\subsubsection{Cylindrical}


The coordinate notation change for the LCCS is given by

\begin{equation}
	\left(\alpha,\beta,\gamma \right) \leftarrow \left(r,\theta,z\right)
\end{equation}



The metric is given by 
\begin{align}
& h_{\alpha} = d\alpha\nonumber\\
& h_{\beta} = \alpha d\beta\\
& h_{\gamma} = d\gamma\nonumber
\end{align}


and the differential line element is
\begin{equation}
	d\mathbf{s} = d\alpha\boldsymbol{\hat{\alpha}} + \alpha d\beta\boldsymbol{\hat{\beta}} + d\gamma\boldsymbol{\hat{\gamma}} 
\end{equation}
which implies a differential arc length of
\begin{equation}
	ds = \sqrt{d\mathbf{s}\cdot d\mathbf{s}} = \sqrt{d\alpha^2 + \alpha^2d\beta^2 + d\gamma^2}
\end{equation}


\begin{table}[h]
	\centering
	\begin{tabular}{|c|c|}
		\hline
		
		plane & area integral \\ \hline 
		$\alpha\beta$ & $A_{\alpha\beta} = \oint_{\partial S} \alpha F_{\alpha} d\beta - F_{\beta}d\alpha$ \\ \hline 
		$\alpha\gamma$ & $A_{\alpha\gamma} = \oint_{\partial S} F_{\gamma}d\alpha - F_{\alpha}d\gamma$ \\ \hline 
		$\beta\gamma$ & $A_{\beta\gamma} = \oint_{\partial S} F_{\beta}d\gamma - \alpha F_{\gamma} d\beta$ \\ \hline 
		
		\hline
	\end{tabular}
	\caption{Area integrals for planar curves in cylindrical geometry}
	\label{t:cylindrical}
\end{table}

\subsubsection{Spherical}

The coordinate notation change for the LCCS is given by

\begin{equation}
	\left(\alpha,\beta,\gamma\right)\leftarrow\left(r,\theta,\phi\right)
\end{equation}




The metric is given by 
\begin{align}
& h_{\alpha} = d\alpha\nonumber\\
& h_{\beta} = \alpha\sin{\left(\gamma\right)}d\beta\\
& h_{\gamma} = \alpha d\gamma\nonumber
\end{align}


and the differential line element is
\begin{equation}
	d\mathbf{s} = d\alpha\boldsymbol{\hat{\alpha}} + \alpha d\gamma\boldsymbol{\hat{\gamma}} + \alpha\sin{\left(\gamma\right)}d\beta\boldsymbol{\hat{\beta}} 
\end{equation}
which implies a differential arc length of
\begin{equation}
	ds = \sqrt{d\mathbf{s}\cdot d\mathbf{s}} = \sqrt{d\alpha^2 + \alpha^2d\gamma^2 + \alpha^2\sin^2{\left(\gamma\right)}d\beta^2}
\end{equation}


\begin{table}[h]
	\centering
	\begin{tabular}{|c|c|}
		\hline
		
		plane & area integral \\ \hline 
		$\alpha\beta$ & $A_{\alpha\beta} = \oint_{\partial S} \alpha\sin{\left(\gamma\right)} F_{\alpha}d\beta - F_{\beta}d\alpha$ \\ \hline 
		$\alpha\gamma$ & $A_{\alpha\gamma} = \oint_{\partial S} F_{\gamma}d\alpha - \alpha F_{\alpha}d\gamma$ \\ \hline 
		$\beta\gamma$ & $A_{\beta\gamma} = \oint_{\partial S} \alpha F_{\beta}d\gamma - \alpha\sin{\left(\gamma\right)}F_{\gamma}d\beta$ \\ \hline 
		
		\hline
	\end{tabular}
	\caption{Area integrals for planar curves in spherical geometry}
	\label{t:spherical}
\end{table}


\subsubsection{Summary of General Area Integrals}


\begin{table}[h]
	\centering
	\begin{tabular}{|c|l|l|l|}
		\hline
		
		plane & Cartesian & Cylindrical & Spherical \\ \hline
		
		$\alpha\beta$ & $\oint F_{\alpha}d\beta - F_{\beta}d\alpha$ & $\oint \alpha F_{\alpha} d\beta - F_{\beta}d\alpha$ & $\oint \alpha\sin{\left(\gamma\right)} F_{\alpha}d\beta - F_{\beta}d\alpha$ \\ \hline
		$\alpha\gamma$ & $\oint F_{\gamma}d\alpha - F_{\alpha}d\gamma$ & $\oint F_{\gamma}d\alpha - F_{\alpha}d\gamma$ & $\oint F_{\gamma}d\alpha - \alpha F_{\alpha}d\gamma$ \\ \hline
		$\beta\gamma$ & $\oint F_{\beta}d\gamma - F_{\gamma}d\beta$ & $\oint F_{\beta}d\gamma - \alpha F_{\gamma} d\beta$ & $\oint \alpha F_{\beta}d\gamma - \alpha\sin{\left(\gamma\right)}F_{\gamma}d\beta$ \\ \hline
		
		\hline
	\end{tabular}
	\caption{Summary of area integrals for the different coordinate systems}
	\label{t:areasummary}
\end{table}

\subsection{Choosing $\vec{\mathbf{F}}$ when $\nabla\cdot\vec{\mathbf{F}}=1$ }

Given the general contour integrals as a function of $\mathbf{F}$ we must now choose the components of the vector field. 

Recall that finding the area of a domain implies that

\begin{equation}
	\nabla\cdot\mathbf{F} = 1
\end{equation}

In order to reduce the number of integrals that need to be calculated, it is best to choose $\mathbf{F}$ such that there is only 1 non-zero component. As we have three coordinate planes, each geometry will need 3 $\mathbf{F}$ vectors. 

\subsubsection{Cartesian}


The divergence in Cartesian coordinates is given by
\begin{equation}
	\nabla\cdot\mathbf{F} = \frac{\partial F_{\alpha}}{\partial \alpha} + \frac{\partial F_{\beta}}{\partial \beta} + \frac{\partial F_{\gamma}}{\partial \gamma}
\end{equation}

This leads to choosing our three $\mathbf{F}$ vectors as:
 
\begin{align}
 \mathbf{F_1} &= \left[\alpha, 0, 0 \right]\\
 \mathbf{F_2} &= \left[0, \beta, 0 \right]\\
 \mathbf{F_3} &= \left[0, 0, \gamma \right]
\end{align}


The area integrals for Cartesian coordinates reduce to


\begin{table}[h]
	\centering
	\begin{tabular}{|c|r|r|r|}
		\hline
		
		plane & $F_{\alpha} = \alpha \left(\mathbf{F_1}\right)$ & $F_{\beta} = \beta \left(\mathbf{F_2}\right)$ & $F_{\gamma} = \gamma \left(\mathbf{F_3}\right)$ \\ \hline
		
		$\alpha\beta$ & $\oint \alpha d\beta$ & $-\oint \beta d\alpha$ & - \\ \hline
		$\alpha\gamma$ & $-\oint \alpha d\gamma$ & - & $\oint \gamma d\alpha$ \\ \hline 
		$\beta\gamma$ & - & $\oint \beta d\gamma$ & $-\oint \gamma d\beta$ \\ \hline
		
		\hline
	\end{tabular}
	\caption{Final area integrals for Cartesian coordinates}
	\label{t:areacartesian}
\end{table}


\subsubsection{Cylindrical}


The divergence in cylindrical coordinates is given by
\begin{equation}
	\nabla\cdot\mathbf{F} = \frac{1}{\alpha}\frac{\partial}{\partial \alpha}\left(\alpha F_{\alpha}\right) + \frac{1}{\alpha}\frac{\partial}{\partial \beta}F_{\beta} + \frac{\partial}{\partial \gamma} F_{\gamma}
\end{equation}

This leads to choosing our three $\mathbf{F}$ vectors as:
 
\begin{align}
 \mathbf{F_1} &= \left[\frac{1}{2}\alpha, 0, 0 \right]\\
 \mathbf{F_2} &= \left[0, \alpha\beta, 0 \right]\\
 \mathbf{F_3} &= \left[0, 0, \gamma \right]
\end{align}


The area integrals for Cylindrical coordinates reduce and are shown in Table~\ref{t:areacylindrical}. 

\begin{table}[h]
	\centering
	\begin{tabular}{|c|r|r|r|}
		\hline
		
		plane & $F_{\alpha} = \frac{1}{2}\alpha \left(\mathbf{F_1}\right)$ & $F_{\beta} = \alpha\beta \left(\mathbf{F_2}\right)$ & $F_{\gamma} = \gamma \left(\mathbf{F_3}\right)$ \\ \hline
		
		$\alpha\beta$ & $\oint \frac{1}{2}\alpha^2 d\beta$ & $-\oint \alpha\beta d\alpha$ & - \\ \hline
		$\alpha\gamma$ & $-\oint \frac{1}{2}\alpha d\gamma$ & - & $\oint \gamma d\alpha$ \\ \hline 
		$\beta\gamma$ & - & $\oint \alpha\beta d\gamma$ & $-\oint \alpha\gamma d\beta$ \\ \hline
		
		\hline
	\end{tabular}
	\caption{Final area integrals for Cylindrical coordinates}
	\label{t:areacylindrical}
\end{table}

At first glance, the integrals in the $\beta\gamma$ plane may seem confusing. If we're in the $\beta\gamma$ plane, how can there be any area if there's no $\alpha$ component? The answer is that we're in \emph{a} $\beta\gamma$ plane ($\alpha=$const), not strictly the $\beta\gamma$ plane corresponding to $\alpha=0$. In physical thinking, the $\beta\gamma$ plane is a tube and we know that the radius ($\alpha$) scales the area of a tube ($A_{tube}=2\pi r\Delta z$). 

\subsubsection{Note on Divergence}
Until now, we've been using the standard 3d divergence in the various coordinate systems to evaluate $\mathbf{F}$, despite evaluating in \emph{planes} in the systems. This has worked so far because the metrics do not change when viewing only a plane in the systems. One must be careful, however, as this is not always true and violation of this will give incorrect results. 

The general divergence equations for an orthogonal coordinate system, given the scale factors, is

\begin{equation}
	\nabla\cdot\mathbf{F} = \frac{1}{h_{\alpha}h_{\beta}h_{\gamma}}\left[\frac{\partial}{\partial \alpha}\left(h_{\beta}h_{\gamma}F_{\alpha}\right)+ \frac{\partial}{\partial \beta}\left(h_{\alpha}h_{\gamma}F_{\beta}\right) + \frac{\partial}{\partial \gamma}\left(h_{\alpha}h_{\beta}F_{\gamma}\right)\right]
\end{equation}

We'll see in the next section on spherical coordinates that this will come into play and we'll have to be more careful about choosing $\mathbf{F}$. 

\subsubsection{Spherical}

The divergence in spherical coordinates is given by
\begin{equation}
	\nabla\cdot\mathbf{F} = \frac{1}{\alpha^2}\frac{\partial}{\partial \alpha}\left(\alpha^2F_{\alpha}\right) + \frac{1}{\alpha\sin{\left(\gamma\right)}}\frac{\partial}{\partial \beta} F_{\beta} + \frac{1}{\alpha\sin{\left(\gamma\right)}}\frac{\partial}{\partial \gamma}\left(\sin{\left(\gamma\right)}F_{\gamma}\right)
\end{equation}

This is, in general, not compatible with our work so far when on planes in $\alpha\beta\gamma$. 

For the $\alpha\beta$ plane:
\begin{equation}
	\nabla_{\alpha\beta}\cdot\mathbf{F} = \frac{1}{\alpha}\frac{\partial}{\partial \alpha}\left(\alpha F_{\alpha}\right) + \frac{1}{\alpha\sin{\left(\gamma\right)}}\frac{\partial}{\partial \beta} F_{\beta} + \frac{1}{\sin{\left(\gamma\right)}}\frac{\partial}{\partial \gamma}\left(\sin{\left(\gamma\right)}F_{\gamma}\right)
\end{equation}

For the $\alpha\gamma$ plane:
\begin{equation}
	\nabla_{\alpha\gamma}\cdot\mathbf{F} = \frac{1}{\alpha}\frac{\partial}{\partial \alpha}\left(\alpha F_{\alpha}\right) + \frac{\partial}{\partial \beta} F_{\beta} + \frac{1}{\alpha}\frac{\partial}{\partial \gamma}F_{\gamma}
\end{equation}


For the $\beta\gamma$ plane:
\begin{equation}
	\nabla_{\beta\gamma}\cdot\mathbf{F} = \frac{1}{\alpha^2}\frac{\partial}{\partial \alpha}\left(\alpha^2 F_{\alpha}\right) + \frac{1}{\alpha\sin{\left(\gamma\right)}}\frac{\partial}{\partial \beta} F_{\beta} + \frac{1}{\alpha\sin{\left(\gamma\right)}}\frac{\partial}{\partial \gamma}\left(\sin{\left(\gamma\right)}F_{\gamma}\right)
\end{equation}


For each plane, we have two possible vectors for a total of 6 vectors to choose. xxx

This leads to choosing our three $\mathbf{F}$ vectors as:
 
\begin{align}
 \mathbf{F_1} &= \left[\frac{1}{2}\alpha, 0, 0 \right]\\
 \mathbf{F_2} &= \left[0, \alpha\beta\sin{(\gamma)}, 0 \right]\\
 \mathbf{F_3} &= \left[0, 0, -\alpha\cot{(\gamma)} \right]
\end{align}

The choice of $F_{\alpha}$ is initially unintuitive if one just attempts to plug into the divergence. In the $r\theta$ and $r\phi$ planes, the first term in the divergence reduces to
\begin{equation}
	\frac{1}{\alpha^2}\frac{\partial}{\partial \alpha}\left(\alpha^2F_{\alpha}\right) \rightarrow \frac{1}{\alpha}\frac{\partial}{\partial \alpha}\left(\alpha F_{\alpha}\right)
\end{equation}
as though we were in cylindrical coordinates, and we lose the multidimensional aspect of radial divergence.

NOTE: There's something weird about the $\beta\gamma$ plane. $F_\beta$ actually needs to be $\beta\sin(\gamma)$, dropping the $\alpha$. If the area factor has its $\Delta r$ taken as zero, do we drop an order of $\alpha$ from all terms?

The area integrals for Spherical coordinates reduce to


\begin{table}[h]
	\centering
	\begin{tabular}{|c|r|r|r|}
		\hline
		
		plane & $F_{\alpha} = \frac{1}{2}\alpha \left(\mathbf{F_1}\right)$ & $F_{\beta} = \alpha\beta\sin{(\gamma)} \left(\mathbf{F_2}\right)$ & $F_{\gamma} = -\alpha\cot{(\gamma)} \left(\mathbf{F_3}\right)$ \\ \hline
		
		$\alpha\beta$ & $\oint \frac{1}{2}\alpha^2\sin{(\gamma)}d\beta$ & $-\oint \alpha\beta\sin{(\gamma)}d\alpha$ & - \\ \hline
		$\alpha\gamma$ & $-\oint \frac{1}{2}\alpha^2d\gamma$ & - & $-\oint \alpha\cot{(\gamma)} d\alpha$ \\ \hline 
		$\beta\gamma$ & - & $\oint \alpha^2\beta\sin{(\gamma)} d\gamma$ & $\oint \alpha^2\cos{(\gamma)} d\beta$ \\ \hline
		
		\hline
	\end{tabular}
	\caption{Final area integrals for Spherical coordinates}
	\label{t:areaspherical}
\end{table}

The same rationale for the presence and interpretation of ``in-plane'' coordinates applies here (see cylindrical coordinates). 



\section{When $\nabla\cdot\vec{\mathbf{F}}$ is a monomial}
\label{s:monomialarea}

Consider now the case when the divergence of $\mathbf{F}$ is a monomial, $\nabla\cdot\mathbf{F} = K\alpha^a\beta^b\gamma^c$. This arises when calculating the volume of a set of faces, or when interpolating quantities between grids. 

\subsection{Cartesian}

For the Cartesian case, our three vectors become

\begin{align}
 \mathbf{F_1} &= \left[\frac{K}{a+1}\alpha^{a+1}\beta^{b}\gamma^{c}, 0, 0 \right]\\
 \mathbf{F_2} &= \left[0, \frac{K}{b+1}\alpha^{a}\beta^{b+1}\gamma^{c}, 0 \right]\\
 \mathbf{F_3} &= \left[0, 0, \frac{K}{c+1}\alpha^{a}\beta^{b}\gamma^{c+1} \right]
\end{align}

The final contour integrals for Cartesian coordinates become

\begin{table}[h]
	\centering
	\begin{tabular}{|c|r|r|r|}
		\hline
		
		plane & $F_{\alpha} = \frac{K}{a+1}\alpha^{a+1}\beta^{b}\gamma^{c} \left(\mathbf{F_1}\right)$ & $F_{\beta} = \frac{K}{b+1}\alpha^{a}\beta^{b+1}\gamma^{c} \left(\mathbf{F_2}\right)$ & $F_{\gamma} = \frac{K}{c+1}\alpha^{a}\beta^{b}\gamma^{c+1} \left(\mathbf{F_3}\right)$ \\ \hline
		
		$\alpha\beta$ & $\oint \frac{K}{a+1}\alpha^{a+1}\beta^{b}\gamma^{c} d\beta$ & $-\oint \frac{K}{b+1}\alpha^{a}\beta^{b+1}\gamma^{c} d\alpha$ & - \\ \hline
		
		$\alpha\gamma$ & $-\oint \frac{K}{a+1}\alpha^{a+1}\beta^{b}\gamma^{c} d\gamma$ & - & $\oint \frac{K}{c+1}\alpha^{a}\beta^{b}\gamma^{c+1} d\alpha$ \\ \hline 
		
		$\beta\gamma$ & - & $\oint \frac{K}{b+1}\alpha^{a}\beta^{b+1}\gamma^{c} d\gamma$ & $-\oint \frac{K}{c+1}\alpha^{a}\beta^{b}\gamma^{c+1} d\beta$ \\ \hline
		
		\hline
	\end{tabular}
	\caption{Final area integrals for Cartesian coordinates}
	\label{t:areacartesianmono}
\end{table}


\subsection{Cylindrical}


For the Cylindrical case, our three vectors become

\begin{align}
 \mathbf{F_1} &= \left[\frac{K}{a+2}\alpha^{a+1}\beta^{b}\gamma^{c}, 0, 0 \right]\\
 \mathbf{F_2} &= \left[0, \frac{K}{b+1}\alpha^{a+1}\beta^{b+1}\gamma^{c}, 0 \right]\\
 \mathbf{F_3} &= \left[0, 0, \frac{K}{c+1}\alpha^{a}\beta^{b}\gamma^{c+1} \right]
\end{align}

The final contour integrals for Cylindrical coordinates become


\begin{table}[h]
	\centering
	\begin{tabular}{|c|r|r|r|}
		\hline
		
		plane & $F_{\alpha} = \frac{K}{a+2}\alpha^{a+1}\beta^{b}\gamma^{c} \left(\mathbf{F_1}\right)$ & $F_{\beta} = \frac{K}{b+1}\alpha^{a+1}\beta^{b+1}\gamma^{c} \left(\mathbf{F_2}\right)$ & $F_{\gamma} = \frac{K}{c+1}\alpha^{a}\beta^{b}\gamma^{c+1} \left(\mathbf{F_3}\right)$ \\ \hline
		
		$\alpha\beta$ & $\oint \frac{K}{a+2}\alpha^{a+2}\beta^{b}\gamma^{c} d\beta$ & $-\oint \frac{K}{b+1}\alpha^{a+1}\beta^{b+1}\gamma^{c} d\alpha$ & - \\ \hline
		
		$\alpha\gamma$ & $-\oint \frac{K}{a+2}\alpha^{a+1}\beta^{b}\gamma^{c} d\gamma$ & - & $\oint \frac{K}{c+1}\alpha^{a}\beta^{b}\gamma^{c+1} d\alpha$ \\ \hline 
		
		$\beta\gamma$ & - & $\oint \frac{K}{b+1}\alpha^{a+1}\beta^{b+1}\gamma^{c} d\gamma$ & $-\oint \frac{K}{c+1}\alpha^{a+1}\beta^{b}\gamma^{c+1} d\beta$ \\ \hline
		
		\hline
	\end{tabular}
	\caption{Final area integrals for Cylindrical coordinates}
	\label{t:areacylindricalmono}
\end{table}


\subsection{Spherical}

For the Spherical case, our 6 vectors become

\begin{align}
 \mathbf{F_{\alpha}^{\alpha\beta}} &= \left[\frac{K}{a+2}\alpha^{a+1}\beta^{b}\gamma^{c}, 0, 0 \right] \\
 \mathbf{F_{\beta}^{\alpha\beta}} &= \left[0, \frac{K}{b+1}\alpha^{a+1}\beta^{b+1}\gamma^{c}, 0 \right] \\
 \mathbf{F_{\alpha}^{\alpha\gamma}} &= \left[\frac{K}{a+2}\alpha^{a+1}\beta^{b}\gamma^{c}, 0, 0 \right] \\
 \mathbf{F_{\gamma}^{\alpha\gamma}} &=  \left[0, 0, \frac{K}{c+1}\alpha^{a+1}\beta^{b}\gamma^{c+1} \right] \\
 \mathbf{F_{\beta}^{\beta\gamma}} &=  \left[0,  \frac{K}{b+1}\alpha^{a+1}\beta^{b+1}\gamma^{c}\sin(\gamma), 0 \right] \\
 \mathbf{F_{\gamma}^{\beta\gamma}} &=  \text{not worth the trouble}
\end{align}


The final contour integrals for Spherical coordinates become

\begin{table}[h]
	\centering
	\begin{tabular}{|c|r|r|r|}
		\hline
		
		plane & $F_{\alpha}^{\text{plane}}$ & $F_{\beta}^{\text{plane}}$ & $F_{\gamma}^{\text{plane}}$ \\ \hline
		
		$\alpha\beta$ & $\oint \frac{K}{a+2}\alpha^{a+2}\beta^{b}\gamma^{c}\sin(\gamma) d\beta$ & $-\oint \frac{K}{b+1}\alpha^{a+1}\beta^{b+1}\gamma^{c}\sin(\gamma) d\alpha$ & - \\ \hline
		
		$\alpha\gamma$ & $-\oint \frac{K}{a+2}\alpha^{a+2}\beta^{b}\gamma^{c} d\gamma$ & - & $\oint \frac{K}{c+1}\alpha^{a+1}\beta^{b}\gamma^{c+1} d\alpha$ \\ \hline 
		
		$\beta\gamma$ & - & $\oint \frac{K}{b+1}\alpha^{a+2}\beta^{b+1}\gamma^{c}\sin(\gamma) d\gamma$ & $-\oint \text{not worth it} d\beta$ \\ \hline
		
		\hline
	\end{tabular}
	\caption{Final area integrals for Spherical coordinates}
	\label{t:areasphericalmono}
\end{table}


\section{Application to Volumetric Quantities}

We will now attempt to build upon the area computations in order to calculate volumetric quantities. Consider a function $g(\alpha,\beta,\gamma)$ defined over some volume $V$. It is possible to write the integral of the quantity as 

\begin{equation}
\label{e:volint}
	\iiint_{V}g(\alpha,\beta,\gamma)dV = \iiint_{V}\nabla\cdot\mathbf{G}dV
\end{equation}

Furthermore, Eq.(\ref{e:volint}) can be rewritten as an area integral via Green's theorem:

\begin{equation}
	\iiint_{V}\nabla\cdot\mathbf{G}dV = \iint_{S}\mathbf{G}\cdot\mathbf{\hat{N}}dA
\end{equation}

But we know that Green's theorem applied in a plane is

\begin{equation}
	\iint_{S}\nabla\cdot\mathbf{F}dA = \oint_{\partial S}\mathbf{F}\cdot\mathbf{\hat{n}}ds
\end{equation}


We then have that 
\begin{equation}
	\nabla\cdot\mathbf{F} = \mathbf{G}\cdot\mathbf{\hat{N}}
\end{equation}

If we consider that $\mathbf{G}$ has only one nonzero component such that $\mathbf{G} = [G_\alpha,0,0]$ and apply the general definition of the divergence (Eq.(\ref{e:divplane})), we arrive at 

\begin{equation}
	\nabla\cdot\mathbf{F}=N_{\alpha}G_{\alpha} = N_{\alpha}\frac{\int h_{\alpha}h_{\beta}h_{\gamma}g(\alpha,\beta,\gamma)d\alpha}{h_{\beta}h_{\gamma}}
\end{equation}



\subsection{If $g(\alpha,\beta,\gamma)$ is a monomial}

Now let us consider the case when $g(\alpha,\beta,\gamma)=C\alpha^\mu\beta^\eta\gamma^\xi$. Then, for each coordinate system, we arrive at 

\begin{equation}
	\nabla\cdot\mathbf{F} = \frac{N_{\alpha}C}{d+1}\alpha^{\mu+1}\beta^\eta\gamma^\xi
\end{equation}

where

\begin{equation}
	d = \begin{cases} 1 &\mbox{if Cartesian}\\
			  2 &\mbox{if Cylindrical}\\
			  3 &\mbox{if Spherical}\end{cases}
\end{equation}

which is a fairly remarkable result given the hassle of the coordinate systems prior to this.


\section{Computing these equations over polyhedra}

\subsection{Integration of polyhedra}

Given a closed surface in $\alpha\beta\gamma$ space composed of planar polygons, we can condense the results of the previous section as the sum of the area integrals over faces

\begin{equation}
	\iiint_V C\alpha^\mu\beta^\eta\gamma^\xi dV = \sum_{f=1}^{N_{faces}} \frac{N_{\alpha}^fC}{d+1} \iint_{S_f} \alpha^{\mu+1}\beta^\eta\gamma^\xi dS_f
\end{equation}

The discussion on the reduction of the surface integral to a line integral was presented in Sec.{\ref{s:monomialarea}.

\subsection{Evaluation of non-planar polygon surface integrals}

Our previous work is only applicable if the surface lies in a plane (not necessarily containing the origin) in $\alpha\beta\gamma$. We can evaluate the surface integral of non-planar polygons by projecting them onto a plane, applying the integrals we have previously derived, and then unprojecting (thereby scaling the resulting integral).

Let $\mathbf{\hat{N}_f}$ be the constant normal to the polygon in $\alpha\beta\gamma$ space. In order to decide which plane to project to, we determine the maximum absolute component $\hat{N}_c$, where $c=\alpha,\beta,\text{ or }\gamma$. We then determine the minimum value of the $c^\text{th}$ component over all vertices (minComp). We can project onto the plane with normal only in the $c^\text{th}$ direction by setting the $c^\text{th}$ component equal to minComp. 

Choosing the maximum absolute component of the normal gives us the largest projection, which helps with numerical errors. One can see this by projecting a planar polygon into one of its orthogonal planes, which gives a line (and results in incorrect answers). 

From here, it is possible to evaluate the surface integral as described in this work.

We can easily rescale the resulting answer as
\begin{equation}
	I_{true} = \frac{|\mathbf{\hat{N}_f}|}{|N_c|}I_{proj}
\end{equation}

This is due to $\alpha\beta\gamma$ being a LCCS, and we therefore do not have to consider effects of the Jacobian. 



\subsection{Integration Over Straight Lines in $\alpha\beta\gamma$}

Given a planar polygon, we can now evaluate the contour integral to determine the surface integral. Let a line in the $\alpha\beta\gamma$ plane connect two points, with the starting vertex index by $v$ and the end vertex indexed by $v+1$. Each coordinate is then given by

\begin{align}
	\alpha(\lambda) &= (1-\lambda)\alpha_{e} + \lambda\alpha_{e+1}\\
	\beta(\lambda) &= (1-\lambda)\beta_{e} + \lambda\beta_{e+1}\\
	\gamma(\lambda) &= (1-\lambda)\gamma_{e} + \lambda\gamma_{e+1}
\end{align}

Consider the integral of two quantities $\mu(\lambda)$ and $\eta(\lambda)$ defined analogously to the above. Then the integral of the multiple of these quantities raised to the powers $p$ and $q$ is

\begin{equation}
	\int_0^1\alpha (\lambda)^p\beta (\lambda)^q\gamma(\lambda)^r d\lambda = \frac{1}{p+q+r+1}\sum_{i=0}^p\sum_{j=0}^q\sum_{k=0}^r \frac{ {p \choose i}  {q \choose j} {r \choose k} }{ {{p+q+r}\choose{i+j+k}}}\alpha ^{i}_{v+1}\alpha ^{p-i}_{v}\beta ^{j}_{v+1}\beta ^{q-j}_{v}\gamma ^{k}_{v+1}\gamma ^{r-k}_{v}
\end{equation}

For the spherical case, a sin/cos term is included in the kernel of the integral, resulting in $\alpha ^p\beta ^q\gamma^r\sin{\gamma}$ or $\alpha ^p\beta ^q\gamma^r\cos{\gamma}$

\begin{align}
\label{e:fullintegrals}
	\int_0^1\alpha (\lambda)^p\beta (\lambda)^q\gamma(\lambda)^r\sin{\gamma}d\lambda &= \sum_{i=0}^p\sum_{j=0}^q\sum_{k=0}^r \frac{ {p \choose i}  {q \choose j} {r \choose k} }{ {{p+q+r}\choose{i+j+k}}}\alpha ^{i}_{v+1}\alpha ^{p-i}_{v}\beta ^{j}_{v+1}\beta ^{q-j}_{v}\gamma ^{k}_{v+1}\gamma ^{r-k}_{v} \int_0^1 B^{p+q+r}_{i+j+k}\left(\lambda\right)\sin{\gamma}d\lambda
%	\int_0^1\alpha (\lambda)^p\beta (\lambda)^q \cos{\gamma}d\lambda &= \sum_{i=0}^p\sum_{j=0}^q \frac{ {p \choose i}  {q \choose j} }{ {{p+q}\choose{i+j}}}\alpha ^{i}_{v+1}\alpha ^{p-i}_{v}\beta ^{j}_{v+1}\beta ^{q-j}_{v} \int_0^1 B^{p+q}_{i+j}\left(\lambda\right)\cos{\gamma}d\lambda
\end{align}
where $B^{p+q+r}_{i+j+k}$ is the $i+j+k$ term of the $p+q+r$ order Bernstein basis. This also appears in the simpler integral above, but reduces to the factor $1/(1+p+q+r)$ that appears infront of the summations. 

The integrals in Eq.(\ref{e:fullintegrals}) can be evaluated as
\begin{align}
%\begin{split}
	\int_0^1 B^{n}_{m}\left(\lambda\right)\sin{\xi}d\lambda = & \pi 2^{-n-3}\Gamma(m+1){n\choose m}\Gamma(-m+n+1) \\ \nonumber
	& \left(4\sin(\xi_{v}) _2\tilde{F}_3\left(\frac{m+1}{2},\frac{m+2}{2};\frac{1}{2},\frac{n+2}{2},\frac{n+3}{2};-\frac{1}{4}(\xi_{v}-\xi_{v+1})^2\right) \right. - \\ 
	& \left. (m+1)(\xi_{v}-\xi_{v+1})\cos(\xi_{v}) _2\tilde{F}_3\left(\frac{m+2}{2},\frac{m+3}{2};\frac{3}{2},\frac{n+3}{2},\frac{n+4}{2};-\frac{1}{4}(\xi_{v}-\xi_{v+1})^2\right)\right) \nonumber
%\end{split}
\end{align}
%\begin{align}
%\begin{split}
%	\int_0^1 B^{n}_{k}\left(\lambda\right)\cos{\gamma}d\lambda = & \pi 2^{-n-3}\Gamma(k+1){n\choose k}\Gamma(-k+n+1) \\ \nonumber
%	& \left(4\cos(\gamma_{v}) _2\tilde{F}_3\left(\frac{k+1}{2},\frac{k+2}{2};\frac{1}{2},\frac{n+2}{2},\frac{n+3}{2};-\frac{1}{4}(\gamma_{v}-\gamma_{v+1})^2\right) \right. + \\ 
%	& \left. (k+1)(\gamma_{v}-\gamma_{v+1})\sin(\gamma_{v}) _2\tilde{F}_3\left(\frac{k+2}{2},\frac{k+3}{2};\frac{3}{2},\frac{n+3}{2},\frac{n+4}{2};-\frac{1}{4}(\gamma_{v}-\gamma_{v+1})^2\right)\right) \nonumber
%\end{split}
%\end{align}

where $_2\tilde{F}_3\left(\vec{a};\vec{b};z\right)$ is the regularized, generalized hypergeometric function.

\end{document}
